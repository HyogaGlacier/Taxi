\documentclass[12pt]{jsarticle}
\usepackage{url}
\usepackage{comment}
\usepackage[top=10truemm,bottom=10truemm,left=10truemm,right=10truemm]{geometry}
\usepackage[dvipdfmx]{graphicx}
\usepackage[dvipdfmx]{color}
\usepackage{ascmac}
\usepackage{amsmath,amssymb}
\usepackage{comment}
\title{Taxi}
\author{fiord}
\begin{document}
	\begin{comment}
	\begin{shadebox}
		\begin{center}
			\textbf{\Huge Taxi}
		\end{center}
		\begin{flushright}
			fiord
		\end{flushright}
	\end{shadebox}
	\hspace{20mm}
	\begin{screen}
	\end{comment}
	\section{冗長な前置き}
		
	%\end{screen}
	
	さて突然ですが、Esolangって聞いたことがありますか?これは難解プログラミング言語のことで、意図的にプログラムが扱いづらいように言語仕様を設定しています。有名どころでは半角スペースとタブ文字、改行だけでコーディングするため画面が真っ白な「Whitespace」やコンパイラが100B前後の「Brain(自主規制)\footnote{自主規制なので分かりづらさ全開ですが察してください。}」といったところでしょうか。他にも、コードを2次元で扱ったりソースが画像になっていたりします。といっても、実際にはただ難解なだけではなく、ネタ性も重視されています。プログラマの一種の遊びで、TSGでも部内大会を開いたりしています。
	
	私はEsolangにはTSGに入ってから触れるようになりました。私が触れた言語の1つに「Taxi」という言語があります。文字通り、タクシーを運転しながら、プログラムを進める、というものです。この言語の特徴として、「意味の通った英語でプログラムを書く」というものがあります。これは実際にコード\footnote{今年の夏に行った第3回コードゴルフ大会で私が書いたものの冒頭です。コード全容は\url{hyoga.hatenablog.com}に落ちています。}を見てもらった方が早いでしょう。
	\begin{table}[h]
		\centering
		\begin{tabular}{|l|}
			\hline
			50 is waiting at Starchild Numerology.\\
			Go to Starchild Numerology: west 1st left, 2nd right, 1st left, 1st left, 2nd left.\\
			Pickup a passenger going to Joyless Park.\\
			Go to Joyless Park: west 1st right, 2nd right, 1st right, 2nd left, 4th right.\\
			Go to Post Office: west 1st left, 1st right, 1st left.\\
			...(この後途方もない量が続きます)\\
			\hline
		\end{tabular}
	\end{table}

	これは英語で実際にタクシーを操作しています。値を設置し、タクシーを移動させて回収、別の場所へと値を動かしています。これもプログラミング言語ですが、一般的に考えられる上に挙げたようなものと比べると、自然言語が扱われており、かなり親しみやすいように感じます。代償としてコードはとても冗長なものになってしまうのですが、私はこの言語を書いていて楽しかったのです(ところで、この時点で普通の人は狂気を感じるのでしょうか。私は普通ですよね…?)。
	
	さて、楽しかった私は、「実際にタクシーを運転してもらうゲーム」という形式でこの言語を知ってもらおうと思ったのです。\\
	
	因みに、私は普段競技プログラミングやCTFを主にやっていますが、特にすごい知識を持っているプロという訳でもなく、両方とも既に部内にプロが居て記事を書くと叩かれそうなので、安全そうなEsolangから出します。が、こちらもプロがいるので、叩かれて灰になっているかもしれません。
	%\newpage
	
	%\begin{screen}
		\section{ゲーム説明}
	%\end{screen}
	
	皆さんにはタクシーを運転してもらいながら、指定された操作(例えば「数字を2つ渡すので、これらの和を求めてください」など)を行ってもらいます。地図上のタクシーを様々な場所へと移動させていくのですが、その際に客を拾って運転していきます。客は1人に1つ値を持っていて、彼らを目的地に下すと、その目的地に応じた演算が行われ、その結果を客として拾うことも出来ます。
	
	さて、簡潔に書こうとしてEsolang並に難解になってしまった説明をした訳ですが、出来る操作は以下の2つです。
	\begin{itemize}
		\item 客を拾う。その際、客の行先を指定する
		\item 移動する。客を乗せていて、その客の目的地に着くと、客は代金を払って降り、演算が行われます。
	\end{itemize}
	Taxi Garageというところに到着すると操作終了で、正しい操作を行ったか判定が行われます。途中でガソリンが無くなる(ガソリンスタンドでお金を払って追加できます)、間違った答えを出力するなどをすると社長にクビにされます。答えが正しければクリア。ランキング機能付きで、「移動距離」と感想に述べている「コード量」の2種類のランキングを用意しています。コード量については、操作中に確定した部分のコードが表示されるので、その中でどうすれば短くなるのか考えていってください。
	
	%\begin{screen}
		\section{感想}
	%\end{screen}

	ゲームで行う内容はTaxi言語に置換できるようになっていて、裏でこっそり言語化を行ってそれをランキングとして用いております。ただ、本来実装されているループを行う際は、そのコードをゲーム中に表示して挿入する、という感じになるしかないかな…と思っています(つまり未実装で、ゲームの内容は全て定数時間で解けるものです)。ただ、そうするとどうしても敷居が上がってしまうので、プログラマでなくても気軽に楽しめる、という前提の下で考えたいです。まあ、実際にはHard以外地図をオリジナルで実装しているので、そのままTaxi言語として動かせる訳ではありません。因みに、逆に任意のTaxiコードをゲーム操作に置き換えることは出来ません。Taxiは途中で操作ミスをすると即クビ=GameOverになってしまうという仕様なので、これをゲームに入れると鬼畜ゲー化してつらいです。
	
	ゲームを製作する中で、結構mnemoからアイデアを引っ張ってきてしまったので、先輩方ごめんなさいという感じが物凄いです。まだデザインを中心に未完成なので、完成していなかったらごめんなさい。大人しくプログラマ展示(制作の終わらなかった人々がその場で実装している様を展示する企画)になっています。完成したとしてもHokkaido Univ.\&Hitachi 1st New-concept Computing Contest 2017があるので、いずれにせよプログラマ展示になっていそうですね$\cdots$
\end{document}